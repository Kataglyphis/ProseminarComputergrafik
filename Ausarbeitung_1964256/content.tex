%% content.tex
%%

%% ================================================================================================================================================================

\chapter{Abstract}

In unser heutigen hoch technologisierten Welt steigt stetig und rasant die Leistungsfähigkeit moderner Grafikhardware.
Um heutzutage Grafik auf einem modernen Computer darzustellen sind mittlerweile eine Vielzahl von Zwischenschritten nötig.
Diese Arbeit beschäftigt sich um die einzelnen Stufen dieser Kette, deren jeweilige Aufgabe, ihren Datentransfer untereinander,
ihre Reihenfolge sowie ihren Arbeitskontext. Beim Arbeitskontext werden unter anderem die unterschiedlichen Koordinatensysteme untersucht. 
Dabei soll nicht nur konkret auf die Arbeitsweise der einzelnen Stufe eingegangen werden, sondern auch im Speziellen auf die Zusammenarbeit und Kommunikation. 
Exemplarische Fragestellungen die behandelt werden sind Folgende: Können in dieser Art der Abarbeitung Flaschenhälse entstehen und 
wie werden Sie umgangen bzw. bekämpft. Welchen Einfluss hat der Programmierer auf die Pipeline bzw. welche Schritte kann er selber
implementieren und welche Schritte werden rein von der Hardware übernommen und können nicht von ihm modifiziert werden. 
Diese Arbeit macht den Aufbau einer modernen Rendering-Pipeline verständlich und vermittelt den Begriff Rasterisierung
Zusätzlich wird es bei dieser Abhandlung, dank des stetigen technologischen Fortschritts, ein Ausblick auf zukünftige Entwicklungen
und Neuerungen gegeben, die zukünftig moderne Rendering-Pipelines beherrschen wird.
%% ================================================================================================================================================================

\chapter{Einleitung}

Wollen wir mit einer Anwendung~\ref{sec:Anwendung} eine Szene von Objekten, z.B. eine Teekanne in einem Stadion, auf einem Computer darstellen, 
so liegt das Objekt zuerst in Form von vielen Eckpunkten(Dreiecken, Linien) in der Anwendung vor. Dabei befinden wir uns im "Model Space".
Wollen wir die Teekanne innerhalb unserer Welt (das Stadion) bewegen, wenden wir auf jede Koordinate des Modells(gemeint sind alle mit dem Modell
assoziierten Eckpunkte, Normalen) den "Model Transform" an und gehen somit in den "World Space". Die geschieht meist in der Anwendung bevor die Geometrie
zur Geometriestufe~\ref{sec:Geometry Stage} geschickt wird. \par
Als Anwender können wir nun zu Beginn der Geometriestufe mit einem eigenen Vertex Shader die Beleuchtungsberechnung etc. der Eckpunkte bestimmen. Dabei
helfen die von der Anwendung mitgegebenen Attribute. Auch in welchen Koordinatensystem dies geschieht bleibt dem Programmierer überlassen. In der anschließenden 
Stufe werden diese Eckpunkte zu Primitiven zusammengesetzt und können weiter unterteilt werden~\ref{subsec:Tessellation}.Die Primitive werden von dem Tessellation- 
zum Geometry Shader durchgereicht, welcher diese vervielfachen, entfernen oder beliebig verändern kann..\par
Um eine (orthographische oder auch perspektivische) Projektion vorzubereiten, wenden wir einen "View Transform" an. Damit können wir in den "View Space" gelangen
und vereinfachen die darauffolgende Projektion. Egal welche Art von Projektion wir angewendet haben, unsere Szene mit der Teekanne im Stadion liegt nun in einem 
Einheitswürfel. \par
Nun kann die Teekanne derart platziert sein, dass Teile außerhalb unseres Sichtfeldes(frustum) liegen. Aufgabe des Clipping~\ref{subsec:Clipping} ist nun
das "Frustum Culling", also das Abschneiden des nicht sichtbaren Bereichs, um weitere Berechnungen innerhalb darauffolgender Stufen zu beschleunigen.
Der Entwickler hat keinen Einfluss auf das Clipping~\ref{subsec:Clipping} (Abschneiden nicht relevanter Szenengeometrie). Ein Algorithmus der dieses Problem löst,
ist der Cohen-Sutherland-Algorithmus.\par
Nach dem Beschneiden liegen nur noch die "sichtbaren" Primitive vor und gelangen in die nächste Pipelinestufe, dem Viewport Transform~\ref{subsec:Viewport Transform}.
Die x-und y-Komponente unserer Eckpunktositionen werden auf die Bildschirmgröße skaliert und bilden die schlussendliche Position des Vertex. Dabei ist es wichtig 
anzumerken, dass wir die z-Komponente für weitere Berechnungen in der Rasterisierung~\ref{sec:Rasterisierung} speichern.\par
Nun sind wir an dem Punkt angelangt, an dem wir unsere transformierten, projizierten, sichtbaren Eckpunkte mit ihren Shadinginformationen vorliegen haben
(Außerdem auch Tiefenwert!). Ziel der Rasterisierung ist es nun jeden einzelnen Bildschirmpixel mit diesen Informationen einzufärben. Vor dem endgültigen Darstellen 
der Bildschirmpixel lassen sich auf den zuvor erzeuten Fragmenten eine Vielzahl von Operationen~\ref{sec:Per Fragment Operations} ausführen.So haben 
wir nun das finale Bild produziert.\par
Besonders interessant sind die neuen Shader für Ray Tracing. Die neuen Einheiten bilden eine Ergänzung der bisherigen Rasterisierung~\ref{sec:Real-Time Raytracing}.
%% ========================================================================================================================================================

\chapter{Moderne Rendering-Pipeline}

    \section{Anwendung}
    \label{sec:Anwendung}
    Zu rendernde Objekte werden zunächst von der Anwendung zur Grafikkarte geschickt. Dabei liegen die Daten über ein
    Objekt beispielhaft im OBJ-Dateiformat vor. Die Informationen über einen Vertex, das sind die Position,
    Normale und Texturkoordinate sind für Berechnungen der nächsten Stufe, des Vertex Shaders~\ref{subsec:Vertex Shader}, wichtig.
    Desweiteren werden Kamera- und Lichtposition in die Geometriestufe~\ref{sec:Geometry Stage} übergeben.
    In der Anwendung können Informationen über ein Objekt geupdatet werden, z.B. bei Kollisionen von Objekten bei denen sich 
    Positionen verändern. Allgemein werden hier Benutzereingaben gesammelt und verarbeitet.
    Als programmierbare Stufe dieser Pipeline kann der Anwender konkrete Verbesserungen in diese Stufe einbringen um z.B. mit einem BVH
    die Anzahl der Primitive für die Geometriestufe zu verrringern.
    
    \begin{figure}[H]
        \centering
        \begin{minipage}[t]{0.45\linewidth}
            \centering
            \includegraphics[width=.75\linewidth]{Bilder/MaterialDefs.PNG}
            \caption{Definition beispielhaftes Material}
        \end{minipage}
        \hfill
        \begin{minipage}[t]{0.45\linewidth}
            \centering
            \includegraphics[width=\linewidth]{Bilder/ObjectData.png}
            \caption{Vertex Daten im OBJ-Dateiformat}
        \end{minipage}
    \end{figure}

    \section{Geometrie Stufe}
    \label{sec:Geometry Stage}

        \subsection{Vertex Shader}
        \label{subsec:Vertex Shader}
            \begin{figure}[H]
                \centering
                \def\svgwidth{\columnwidth}
                \import{Bilder/}{VertexShader.pdf_tex}
                \label{Vertex Shader}
                \caption{Funktionsweise Vertex Shader}
            \end{figure}

        Hier arbeiten wir komplett in Objektkoordinaten.

        \subsection{Primitive Assembly}
        \label{subsec:Primitive Assembly}

        \subsection{Tessellation}
        \label{subsec:Tessellation}

            \begin{figure}[H]
                \centering
                \def\svgwidth{\columnwidth}
                \import{Bilder/}{Tessellation.pdf_tex}
                \label{Tessellation Shader}
                \caption{Funktionsweise Tessellation Shader}
            \end{figure}

        Hinzufügen neuer Vertices.

        \subsection{Geometry Shader}
        \label{subsec:Geometry Shader}

            \begin{figure}[H]
                \centering
                \def\svgwidth{\columnwidth}
                \import{Bilder/}{GeometryShader.pdf_tex}
                \label{Geometry Shader}
                \caption{Funktionsweise Geometry Shader}
            \end{figure}

        \subsection{Clipping}
        \label{subsec:Clipping}
        Um Objekte bzw. Objektausschnitte, welche außerhalb des Sichtfensters liegen, für die Bildsynthese zu verwerfen kommt nun das Abschneiden(englisch = "clipping").
        Algorithmus von Sutherland-Hodgman.
        Wir gehen vom Clip-Space zum Window-Space

        \subsection{Viewport Transform}
        \label{subsec:Viewport Transform}

    \section{Rasterisierung}
    \label{sec:Rasterisierung}
        \begin{figure}[H]
            \centering
            \def\svgwidth{\columnwidth}
            \import{Bilder/}{Rasterisierung.pdf_tex}
            \label{Rasterisierung}
            \caption{Ablauf Rasterisierung}
        \end{figure}

    In der vorherigen Geometriestufe wurde einem klar, dass wir ein Objekt nach dem Anderen betrachten (Object-ordered Rendering) 
    
    \section{Fragment Shader}
    \label{sec:Fragment Shader}

    \section{Per Fragment Operations}
    \label{sec:Per Fragment Operations}

        \subsection{Multisample Fragment Ops}
    
        \subsection{Stencil Test}

        \subsection{Occlusion Query}
    
        \subsection{Blending}

        \subsection{Logical Operations}

    \section{Framebuffer und Buffer Objekte}
    \label{sec:Framebuffer}

    \section{GPU Memory}
    \label{sec:GPU Memory}

    \section{Compute Shader}
    \label{sec:Compute Shader}

    \begin{figure}[H]
        \centering
        \def\svgwidth{\columnwidth}
        \import{Bilder/}{ComputeShader.pdf_tex}
        \label{Compute Shader}
        \caption{Funktionsweise Compute Shader}
    \end{figure}

    Mit der rasant steigenden Leistung heutiger Grafikhardware stieg auch der Wunsch beim Anwender nach mehr Rechenleistung.
    Parallesisierbare Arbeit wird in Threadgroups eingeteilt. Threads innerhalb einer Gruppe laufen gleichzeitig, wohingegen
    die Threadgruppen untereinander dies nicht müssen.

\chapter{Unterschied moderne und klassische Rendering-Pipeline}

    \section{Freie Programmierung oder reines Konfigurieren}
        
        \begin{figure}[H]
            \centering
            \begin{minipage}[t]{0.45\linewidth}
                \centering
                \def\svgwidth{\columnwidth}
                \import{Bilder/}{AltesOpenGL.pdf_tex}
                \label{Altes OpenGL}
                \caption{OpneGL $\leq$ 2.0}
            \end{minipage}
            \hfill
            \begin{minipage}[t]{0.45\linewidth}
                \centering
                \def\svgwidth{\columnwidth}
                \import{Bilder/}{ModernesOpenGL.pdf_tex}
                \label{Modernes OpenGL}
                \caption{Modernes OpenGL}
            \end{minipage}
        \end{figure}

    \section{Vertex Arrays, Index Buffers, ..}

%% ===============================================================================================================================
\chapter{Ausblick}

\section{Raytracing Unterstützung}
\label{sec:Real-Time Raytracing}
In heutigen modernen Grafikprogrammierschnittstellen(Vulkan, DirectX) befindet sich ray-tracing
Moderne Ansätze gehen von der objektbasierten (siehe Rasterisierung) zu der Image-ordered Bilderstellung über.
Hiermit werden nicht nur Primär- sondern auch Sekundärstrahlen(etc.) betrachtet und somit unter Anderem Schatten und Spiegelungen
erreicht ohne dafür spezielle Techniken verwenden zu müssen (siehe Rasterisierung). 

\section{Task-/Mesh Shaders}
\label{sec:Meshshader}
Allgemein lässt sich an dieser neuen Technik der Trend von fixen Vorgängen innerhalb der Pipeline zu mehr
flexibleren wiedererkennen. So kann der Task Shader mit der Control Shader Einheit der bisherigen Tessellation verglichen werden. 
Jedoch arbeiten wir hier mit mehreren Threads und mit frei wählbaren In- und Output! Mehr Flexibilität ist 
die Devise.
Anstatt einzelne Dreiecke einzeln zu berechnen wie in der bisherigen Pipeline können wir mehrere gleichzeitig
bearbeiten in sogenannten parallelen Thread-Gruppen. Hier wird also nichts anderes als das Modell von Compute Shader
genommen und auf die Grafikpipeline angewandt. Vorallem bei komplexen Szenen mit vielen Millionen Dreicken 
verspricht man sich eine Entlastung der CPU. Durch diesen Aufbau können wir bereits im Vorfeld viele Dreiecke vom Rendern 
ausschließen. \par
Dabei wird ein komplexes Mesh in einzelne "Meshlets" zerlegt, welche wiederrum von Mesh Shadern behandelt werden.

\includegraphics[width=1.\textwidth]{Bilder/meshlets_pipeline.png}
%% ========================================================================================================================================


%% content.tex
%%
