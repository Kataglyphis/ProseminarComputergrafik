%% content.tex
%%

%% ==============================

\chapter{Abstract}
In unser heutigen hoch technologisierten Welt steigt stetig und rasant die Leistungsfähigkeit moderner Hardware.
So auch die aktueller Grafikhardware. Einhergehend ist damit allerdings auch steigende Komplexität, die mit 
neuen Konzepten und Ideen in die Technologie einfließt. Um heutzutage Grafik auf einem Computer darzustellen 
sind viele verschiedene Schritte nötig, die hintereinander bzw. gleichzeitig ablaufen. 
Diese Arbeit beschäftigt sich genau um diese Schritte, also deren genaue Spezifizierung, welcher Schritt folgt auf diesen Schritt,
welche Schritte können parallel abgearbeitet werden. Dabei soll nicht nur konkret auf die Arbeitsweise der einzelnen Stufe eingegangen
werden, sondern auch im Speziellen auf die Zusammenarbeit und Kommunikation. 
Exemplarische Fragestellungen die behandelt werden sind Folgende: Können in dieser Art der Abarbeitung Flaschenhälse entstehen und 
wie werden Sie umgangen bzw. bekämpft. Welchen Einfluss hat der Programmierer auf die Pipeline bzw. welche Schritte kann er selber
implementieren und welche Schritte werden rein von der Hardware übernommen und können nicht von ihm modifiziert werden. 
Nach dem Lesen dieser Arbeit wird Ihnen der Aufbau einer modernen Rendering-Pipeline verständlich sein.
%% ==============================
\chapter{Einleitung}
\label{ch:Introduction}

%% ==============

\chapter{Rasterisierungs-Pipeline}

\section{blub}

\dots

\section{blub}

\dots

\section{blub}

\dots

%% ===========================
\chapter{Moderne Rendering-Pipeline}
\label{ch:Content1:sec:Section1}
%% ===========================


%% content.tex
%%

\dots
